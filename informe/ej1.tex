\section{Ejercicio 1}
En el primer ejercicio se pide implementar el método push\_front de la lista atómica. Aquí tenemos crear un nuevo nodo y que este sea la nueva cabeza de la lista y que apunte a la vieja cabeza. El primer problema que nos cruzamos fue que primero realizabamos un exchange atómico de la cabeza antigua con el nuevo nodo y luego se lo apuntaba la la nueva cabeza. Este conjunto de operaciones no era atómico por lo que en ciertas situaciones si justo se intentaba de iterar la lista en el medio de estas dos operaciones habría una condición de carrera en la que la lista podría contener un solo elemento. Esto se solucionó aplicando la función atómica compare\_exchange\_weak a la vieja cabeza de la lista, asegurando que el nuevo nodo siempre apunte al último nodo agregado exitosamente.
